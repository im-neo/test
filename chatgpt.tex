以下是各题的解答步骤和答案:

**1.** \[ \left\{ \begin{aligned} x &= \ln \sqrt{1+t^2} \\ y &= \arctan t \end{aligned} \right. \]
求 \(\frac{dy}{dx}\) 和 \(\frac{d^2y}{dx^2}\)。

首先,求出 \(\frac{dx}{dt}\) 和 \(\frac{dy}{dt}\):
\[ x = \ln \sqrt{1+t^2} = \frac{1}{2} \ln (1+t^2) \Rightarrow \frac{dx}{dt} = \frac{1}{2} \cdot \frac{2t}{1+t^2} = \frac{t}{1+t^2} \]
\[ y = \arctan t \Rightarrow \frac{dy}{dt} = \frac{1}{1+t^2} \]

因此:
\[ \frac{dy}{dx} = \frac{\frac{dy}{dt}}{\frac{dx}{dt}} = \frac{\frac{1}{1+t^2}}{\frac{t}{1+t^2}} = \frac{1}{t} \]

接下来,求 \(\frac{d^2y}{dx^2}\):
\[ \frac{d}{dt}\left( \frac{dy}{dx} \right) = \frac{d}{dt}\left( \frac{1}{t} \right) = -\frac{1}{t^2} \]
\[ \frac{d^2y}{dx^2} = \frac{\frac{d}{dt}\left( \frac{dy}{dx} \right)}{\frac{dx}{dt}} = \frac{-\frac{1}{t^2}}{\frac{t}{1+t^2}} = -\frac{1+t^2}{t^3} \]

所以,答案是:
\[ \frac{dy}{dx} = \frac{1}{t} \]
\[ \frac{d^2y}{dx^2} = -\frac{1+t^2}{t^3} \]

**2.** \[ \left\{ \begin{aligned} x &= 2e^t \\ y &= e^t \end{aligned} \right. \]
求 \(\frac{dy}{dx}\) 和 \(\frac{d^2y}{dx^2}\)。

首先,求出 \(\frac{dx}{dt}\) 和 \(\frac{dy}{dt}\):
\[ x = 2e^t \Rightarrow \frac{dx}{dt} = 2e^t \]
\[ y = e^t \Rightarrow \frac{dy}{dt} = e^t \]

因此:
\[ \frac{dy}{dx} = \frac{\frac{dy}{dt}}{\frac{dx}{dt}} = \frac{e^t}{2e^t} = \frac{1}{2} \]

接下来,求 \(\frac{d^2y}{dx^2}\):
\[ \frac{d}{dt}\left( \frac{dy}{dx} \right) = \frac{d}{dt}\left( \frac{1}{2} \right) = 0 \]
\[ \frac{d^2y}{dx^2} = \frac{0}{\frac{dx}{dt}} = 0 \]

所以,答案是:
\[ \frac{dy}{dx} = \frac{1}{2} \]
\[ \frac{d^2y}{dx^2} = 0 \]

**3.** \[ \left\{ \begin{aligned} x &= a \cos^3 t \\ y &= a \sin^3 t \end{aligned} \right. \]
求 \(\frac{dy}{dx}\) 和 \(\frac{d^2y}{dx^2}\)。

首先,求出 \(\frac{dx}{dt}\) 和 \(\frac{dy}{dt}\):
\[ x = a \cos^3 t \Rightarrow \frac{dx}{dt} = 3a \cos^2 t \cdot (-\sin t) = -3a \cos^2 t \sin t \]
\[ y = a \sin^3 t \Rightarrow \frac{dy}{dt} = 3a \sin^2 t \cos t \]

因此:
\[ \frac{dy}{dx} = \frac{\frac{dy}{dt}}{\frac{dx}{dt}} = \frac{3a \sin^2 t \cos t}{-3a \cos^2 t \sin t} = -\frac{\sin t}{\cos t} = -\tan t \]

接下来,求 \(\frac{d^2y}{dx^2}\):
\[ \frac{d}{dt}\left( \frac{dy}{dx} \right) = \frac{d}{dt}\left( -\tan t \right) = -\sec^2 t \]
\[ \frac{d^2y}{dx^2} = \frac{-\sec^2 t}{\frac{dx}{dt}} = \frac{-\sec^2 t}{-3a \cos^2 t \sin t} = \frac{\sec^2 t}{3a \cos^2 t \sin t} = \frac{1}{3a \sin t \cos^4 t} \]

所以,答案是:
\[ \frac{dy}{dx} = -\tan t \]
\[ \frac{d^2y}{dx^2} = \frac{1}{3a \sin t \cos^4 t} \]

**4.** 求函数 \( f(x, y) = y^3 - x^2 + 6x - 12y \) 的极值。

求偏导数:
\[ \frac{\partial f}{\partial x} = -2x + 6 \]
\[ \frac{\partial f}{\partial y} = 3y^2 - 12 \]

令偏导数为零,解方程组:
\[ -2x + 6 = 0 \Rightarrow x = 3 \]
\[ 3y^2 - 12 = 0 \Rightarrow y^2 = 4 \Rightarrow y = \pm 2 \]

所以,候选点为 \((3, 2)\) 和 \((3, -2)\)。

计算二阶偏导数:
\[ \frac{\partial^2 f}{\partial x^2} = -2 \]
\[ \frac{\partial^2 f}{\partial y^2} = 6y \]
\[ \frac{\partial^2 f}{\partial x \partial y} = 0 \]

计算判别式 \(D\):
\[ D = \frac{\partial^2 f}{\partial x^2} \cdot \frac{\partial^2 f}{\partial y^2} - \left( \frac{\partial^2 f}{\partial x \partial y} \right)^2 = (-2)(6y) - 0 = -12y \]

在 \((3, 2)\) 处:
\[ D = -12 \times 2 = -24 < 0 \]
故 \((3, 2)\) 为鞍点。

在 \((3, -2)\) 处:
\[ D = -12 \times (-2) = 24 > 0 \]
且 \(\frac{\partial^2 f}{\partial x^2} = -2 < 0\),故 \((3, -2)\) 为极大值点。

所以,函数 \(f(x, y)\) 在 \((3, -2)\) 处有极大值。

**5.** 求函数 \( g(x, y) = 4(x - y) - x^2 - y^2 \) 的极值。

求偏导数:
\[ \frac{\partial g}{\partial x} = 4 - 2x \]
\[ \frac{\partial g}{\partial y} = -4 - 2y \]

令偏导数为零,解方程组:
\[ 4 - 2x = 0 \Rightarrow x = 2 \]
\[ -4 - 2y = 0 \Rightarrow y = -2 \]

所以,候选点为 \((2, -2)\)。

计算二阶偏导数:
\[ \frac{\partial^2 g}{\partial x^2} = -2 \]
\[ \frac{\partial^2 g}{\partial y^2} = -2 \]
\[ \frac{\partial^2 g}{\partial x \partial y} = 0 \]

计算判别式 \(D\):
\[ D = \frac{\partial^2 g}{\partial x^2} \cdot \frac{\partial^2 g}{\partial y^2} - \left( \frac{\partial^2 g}{\partial x \partial y} \right)^2 = (-2)(-2) - 0 = 4 > 0 \]
且 \(\frac{\partial^2 g}{\partial x^2} = -2 < 0\),故 \((2, -2)\) 为极大值点。

所以,函数 \(g(x,

 y)\) 在 \((2, -2)\) 处有极大值。

**6.** 设 \(D: \{ (x, y) \mid x \geq 0, y \geq 0, x^2 + y^2 \leq 1 \} \),且
\[ f(x, y) = 1 - 2xy + 8(x^2 + y^2) \]
求 \(\iint_D f(x, y) dxdy\)。

将 \(f(x, y)\) 带入双重积分:
\[ \iint_D f(x, y) dxdy = \iint_D (1 - 2xy + 8(x^2 + y^2)) dxdy \]

由于 \(D\) 是单位圆的第一象限部分,使用极坐标变换:
\[ x = r \cos \theta \]
\[ y = r \sin \theta \]
\[ dxdy = r dr d\theta \]

积分区域 \(D\) 在极坐标下为 \(0 \leq r \leq 1\), \(0 \leq \theta \leq \frac{\pi}{2}\):

\[
\begin{aligned}
&\iint_D f(x, y) dxdy \\
&= \int_0^{\frac{\pi}{2}} \int_0^1 \left(1 - 2(r \cos \theta)(r \sin \theta) + 8(r^2 \cos^2 \theta + r^2 \sin^2 \theta)\right) r dr d\theta \\
&= \int_0^{\frac{\pi}{2}} \int_0^1 \left(1 - 2r^2 \cos \theta \sin \theta + 8r^2 (\cos^2 \theta + \sin^2 \theta)\right) r dr d\theta \\
&= \int_0^{\frac{\pi}{2}} \int_0^1 \left(1 - 2r^2 \cos \theta \sin \theta + 8r^2\right) r dr d\theta \\
&= \int_0^{\frac{\pi}{2}} \left( \int_0^1 r dr - \int_0^1 2r^3 \cos \theta \sin \theta dr + \int_0^1 8r^3 dr \right) d\theta \\
&= \int_0^{\frac{\pi}{2}} \left( \frac{1}{2} - \frac{2}{4} \cos \theta \sin \theta + 2 \right) d\theta \\
&= \int_0^{\frac{\pi}{2}} \left( \frac{1}{2} + 2 - \frac{1}{2} \sin 2\theta \right) d\theta \\
&= \left(\frac{1}{2} + 2\right) \frac{\pi}{2} - \frac{1}{2} \int_0^{\frac{\pi}{2}} \sin 2\theta d\theta \\
&= \frac{5\pi}{4} - \frac{1}{4} \left[ -\cos 2\theta \right]_0^{\frac{\pi}{2}} \\
&= \frac{5\pi}{4} - \frac{1}{4} (1 - (-1)) \\
&= \frac{5\pi}{4} - \frac{1}{2} \\
&= \frac{5\pi}{4} - \frac{2}{4} \\
&= \frac{5\pi}{4} - \frac{1}{2} \\
&= \frac{5\pi - 2}{4}
\end{aligned}
\]

**7.** 求由抛物线 \(y = x^2\) 与直线 \(y = 2x + 3\) 所围平面图形的面积。

求交点:
\[ x^2 = 2x + 3 \]
\[ x^2 - 2x - 3 = 0 \]
\[ (x - 3)(x + 1) = 0 \]
\[ x = 3 \quad \text{或} \quad x = -1 \]

积分求面积:
\[ \int_{-1}^3 (2x + 3 - x^2) dx \]
\[
= \left[ x^2 + 3x - \frac{x^3}{3} \right]_{-1}^3 
= \left(9 + 9 - 9 - (-1 + (-3) - \frac{-1}{3}) \right)
= \left(0 - (-4 + \frac{1}{3})\right)
= \left(4 - \frac{1}{3}\right)
= \frac{11}{3}
\]

**8.** \[ \iiint_G xyzdV \],其中 \(G\) 是平面 \(x + y + z = 1\) 与坐标面所围成的四面体。

四面体顶点为 \((1,0,0)\),\((0,1,0)\),\((0,0,1)\) 和 \((0,0,0)\)。

积分范围:
\[ 0 \leq x \leq 1 \]
\[ 0 \leq y \leq 1 - x \]
\[ 0 \leq z \leq 1 - x - y \]

\[
\iiint_G xyzdV = \int_0^1 \int_0^{1-x} \int_0^{1-x-y} xyz \, dz \, dy \, dx 
\]

先对 \(z\) 积分:
\[
\int_0^{1-x-y} z \, dz = \left[ \frac{z^2}{2} \right]_0^{1-x-y} = \frac{(1-x-y)^2}{2}
\]

对 \(y\) 积分:
\[
\int_0^{1-x} xy \cdot \frac{(1-x-y)^2}{2} \, dy 
\]

设 \(u = 1-x-y\),则 \(-du = dy\),当 \(y = 0\),\(u = 1-x\);当 \(y = 1-x\),\(u = 0\):

\[
\int_{1-x}^0 x(1-x-u) \cdot \frac{u^2}{2} (-du) = \int_0^{1-x} x (1-x-u) \cdot \frac{u^2}{2} du
\]

分开积分:

\[
\int_0^{1-x} x (1-x-u) \cdot \frac{u^2}{2} du = x \int_0^{1-x} \frac{u^2}{2} du - x \int_0^{1-x} \frac{u^3}{2} du
\]

计算:

\[
\int_0^{1-x} \frac{u^2}{2} du = \left[ \frac{u^3}{6} \right]_0^{1-x} = \frac{(1-x)^3}{6}
\]

\[
\int_0^{1-x} \frac{u^3}{2} du = \left[ \frac{u^4}{8} \right]_0^{1-x} = \frac{(1-x)^4}{8}
\]

最终积分结果:

\[
x \left( \frac{(1-x)^3}{6} - \frac{(1-x)^4}{8} \right)
\]

对 \(x\) 积分:

\[
\int_0^1 x \left( \frac{(1-x)^3}{6} - \frac{(1-x)^4}{8} \right) dx
\]

\[
\int_0^1 \frac{x(1-x)^3}{6} dx - \int_0^1 \frac{x(1-x)^4}{8} dx
\]

分别积分:

\[
\frac{1}{6} \int_0^1 x(1-3x+3x^2-x^3) dx - \frac{1}{8} \int_0^1 x(1-4x+6x^2-4x^3+x^4) dx
\]

计算:

\[
\frac{1}{6} \left( \left[ \frac{x^2}{2} - \frac{3x^3}{3} + \frac{3x^4}{4} - \frac{x^5}{5} \right]_0^1 \right) - \frac{1}{8} \left( \left[ \frac{x^2}{2} - \frac{4x^3}{3} + \frac{3x^4}{2} - \frac{x^5}{2} + \frac{x^6}{6} \right]_0^1 \right)
\]

代入:

\[
\frac{1}{6} \left( \frac{1}{2} - 

1 + \frac{3}{4} - \frac{1}{5} \right) - \frac{1}{8} \left( \frac{1}{2} - \frac{4}{3} + \frac{3}{2} - \frac{1}{2} + \frac{1}{6} \right)
\]

简化:

\[
\frac{1}{6} \left( \frac{10}{20} - \frac{20}{20} + \frac{15}{20} - \frac{4}{20} \right) - \frac{1}{8} \left( \frac{10}{20} - \frac{26.67}{20} + \frac{30}{20} - \frac{10}{20} + \frac{3.33}{20} \right)
\]

最终结果:

\[
\frac{1}{6} \left( \frac{1}{5} \right) - \frac{1}{8} \left( \frac{1}{6} \right)
\]

**9.** 将 \( f(x) = \arctan x \) 展开成 \(x\) 的幂级数。

\[
f(x) = \arctan x = \sum_{n=0}^{\infty} (-1)^n \frac{x^{2n+1}}{2n+1}
\]

**10.** 将 \( f(x) = \ln (x^2 + 5x + 4) \) 展开成 \(x\) 的幂级数。

先因式分解:

\[
f(x) = \ln (x^2 + 5x + 4) = \ln ((x+1)(x+4)) = \ln (x+1) + \ln (x+4)
\]

使用级数展开:

\[
\ln (1+x) = x - \frac{x^2}{2} + \frac{x^3}{3} - \cdots
\]

令 \(x\) 为对应值,合并即为所求级数展开。